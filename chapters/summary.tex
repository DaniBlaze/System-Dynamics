The main objective of this paper is analyzing waste sorting behavior and the different factors which influence waste sorting behavior in practice. The reasoning behind this is to help Oslo municipality with reaching a material recycling rate of 50\% by 2030. 

\indent \newline 
The target group consists of 20-39 year old's, not coming from Asia, Africa, Latin-America and Eastern Europe outside of EU. This represents approximately 27\% of all Oslo citizens, and is adjusted for in the simulation model. This means that the model will not be able to reach a realistic material recycling rate of 50\%.

\indent \newline 
A causal loop diagram is presented in order to give an overview of the different variables and their relationships. This includes both positive and negative links on waste sorting behavior. 

\indent \newline 
The simulation model was developed through expanding the existing stocks and flows model with findings from the causal loop diagram. Five initiatives are displayed in the model and consists of increasing advertising effectiveness, increasing the number of waste containers (new quality), implementing smart waste containers (improvements), utilizing economic incentives and increasing frequency of waste collection. The simulations show a "base case" scenario, with a resulting material recycling rate of 31.69\%, while the top three initiatives are increasing the number of waste containers, increasing frequency of waste collection and implementing smart waste containers, with a respectively recycling rate of 36.53\%, 34.4\% and 33.86\%.

\indent \newline 
A sensitivity analysis was performed in order to test the initiatives in different scenarios, which did not the change the recommended suggestions. 