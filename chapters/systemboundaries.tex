During the last decade, Oslo has been striving to be the next world leading environmentally friendly city when it comes to household recycling. Oslo's main focus is to sustain resources and to achieve a circle-based waste management system. The municipality has for many years been struggling to reach their main goals. In 2006, they set a goal of a 50\% recycling rate by 2014, which evidently turned out unsuccessful \cite{effect}. Oslo municipality was forced to rethink their strategy and extend their goals for further improvements. Oslo municipality now has a goal to reach a 50\% recycling rate for households before 2030, which seems more achievable. This has been deemed an enormous challenge, as there is a stagnating recycling rate in the municipality of Oslo. Today, Oslo has a stable recycling rate of 37\% \cite{oslokommune}.

\indent \newline 
The existing system utilizes three types of garbage bags; a blue bag for plastic, a green bag for food waste and regular plastic bags for rest waste. There is a waste sorting system in-house, which lets the consumer systematically organize their own garbage bags. Regarding curbside collection, there are three different containers; one for plastic and food waste, one for paper/cardboard, and one for rest waste. The frequency of collected waste depends on the area of collection. 

\indent \newline
The main objective of this paper is to analyze the waste sorting behavior of the younger population in Oslo. More specifically, the focus will be on the age group 20-39 year olds and not coming from Asia, Africa, Latin-America and Eastern Europe outside of EU. This particular target group is one of the weakest in terms of household recycling \cite[p. 38]{Mikkelborg}. There are several underlying reasons for this. For instance, this target group often lives in household collectives, which usually consists of a broad mix of people, with different lifestyles and perspectives. If the household lacks cohesiveness, it is very hard to sustain a good recycling behavior. Even though half the household recycles, it would quickly get "un-sorted" by the other half. Secondly, there is a cost associated with waste sorting. Many students and young people have a hard time being financially stable, and any additional household costs would impact their daily life. Finally, the target groups' waste sorting behavior is influenced by time and knowledge. It takes a couple of minutes every day to sort the household waste, which seems small at first, but adds up to a lot. The garbage bags are smaller, which results in taking out the thrash more often. In addition, it is common behavior for this target group to be pessimistic to certain ideas. "It wouldn’t be a noticeable difference if we sort our waste" is a frequent phrase for this target group. There are also several answers to questions they are unaware of, such as the frequency of waste collection and environmental impact of waste sorting. 

\indent \newline
There is, however, a huge potential to improve this target group. With the correct communication, advertisement and incentives through efforts from Oslo municipality, this target group can potentially help Oslo municipality reach their goal within 2030. With a suitable approach, this group can be turned around, which can potentially lead to a higher achieved recycling rate for households in Oslo. 

\indent \newline
The following chapters explains the different variables and factors which influence waste sorting behavior in practice. This includes presenting a causal loop diagram, a stock and flow model, before different initiatives and changes are suggested, in order for Oslo municipality to reach the desired material recycling rate. 

